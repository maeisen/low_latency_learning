The increasing scale of modern IoT and industrial control systems has motivated the development of wireless control system technology that can achieve reliable performance in these settings~\cite{varghese2014wireless, li2017review}. One of the primary challenge of wireless control in industrial settings, however, is the time sensitive nature of the systems, thus requiring low latency wireless transmissions~\cite{varghese2014wireless}.  The noise of the wireless channel makes it difficult to simultaneously maintain high reliability while achieving low latency. This motivates the design of resource allocation and scheduling strategies that can both meet reliability \emph{and} latency requirements of the industrial control system.


In the wireless communications research and industry, many radio resource allocation schemes in the form of wireless scheduling techniques have been proposed to provide reliability, or quality of service (QoS), to users across the network in the form of throughput, fairness and/or latency~\cite{cao2001scheduling,wongthavarawat2003packet,fattah2002overview,yaacoub2012survey}. For time-sensitive applications, delay-aware schedulers such as EDF~\cite{wu2014analysis} and WFQ~\cite{lu1999fair} have been developed,  while M-LWDF \cite{andrews2001providing} extends these ideas to include channel state information.


Likewise, in the context of wireless control systems, dynamic schedulers have been developed that provide access to the communication medium dynamically at each step. %, for example by dynamically assigning priorities to the competing tasks.
Initial approaches make scheduling decisions based on abstractions of control performance~\cite{Branicky_RM, Liu_Real_time_systems}. More recently, ``control-aware'' scheduling approaches make decisions explicitly based on current control system states ,~\cite{Donkers_switched, Walsh_stability, Hristu_Kumar_interrupt_based, Egerstedt_queue, Hirche_Scheduling_Price, Cervin_event_scheduling, mamduhi2014event,shi2011optimal,han2017optimal}.
%
Further work takes into account current wireless channel conditions when attempting to meet target control system reliability requirements~\cite{GatsisEtal15}. The work here takes a similar channel-based opportunistic approach, while further incorporating current control states with the explicit goal of meeting low-latency requirements. The proposed approach has been developed specifically for the IEEE 802.11ax protocol in \cite{EisenEtal18}. 



%In this paper, we develop a control-aware, high reliability, low-latency IEEE 802.11ax WiFi protocol~\cite{bellalta2016ieee} that is designed to reduce the total transmission time at each uplink cycle in the control loop. This is done through the mathematical formulation of the control system design goal in the form of a Lyapunov function that ensures stability of the control system. This formulation naturally induces a bound on the packet delivery rate each control system needs to achieve to meet the control-based goal. Such packet delivery rates depend upon current control and channel states and thus dynamically change over the course of the system life time. This can be viewed as an opportunistic protocol with respect to both the current control states \emph{and} channel conditions. Furthermore, because these control-based success rate requirements may be significantly lower than traditional, high reliability communication demands, the proposed method is better suited to find scheduling configurations that can moreover meet the strict latency requirements imposed by the physical system. This is in contrast to the control-aware approach taken in \cite{GatsisEtal15}, which focuses on low-power and infrequent transmissions rather than the low-latency setting of interest in industrial control.

This paper is organized as follows. We formulate the wireless control system in which state information is communicated to the control over a wireless channel as a switched dynamical system (Section \ref{sec_problem}). We then discuss the communication architecture that determines the speed and error rate of transmissions (Section \ref{sec_comm_model}). With this formulation, we adapt concepts of control-communication control-aware for low latency settings by using current control states and channel conditions to derive dynamic packet success rates necessary for each user (Section \ref{sec_codesign}). In this manner, control systems with the most critical communication needs are given higher packet delivery rate targets to meet. The scheduling procedure leverages these dynamic, more liberal rate requirements to reduce total latency, incorporating both a selective scheduling procedure  (Section \ref{sec_rss}) and an assignment-method based that attempts to further reduce total transmission time (Section \ref{sec_assignment}). The performance of the control-aware method is analyzed in a representative low-latency simulation experiment in which its performance is compared against a control-agnostic procedure (Section \ref{sec_numerical_results}). 

