\documentclass[11pt]{article}
\usepackage{pgfplots,graphics, subfigure,vmargin, amsfonts, amsbsy, amsmath, color}
\usepackage{for_authors_reply}
\input{mysymbol.sty}

\setpapersize{USletter}
%\setmarginsrb{leftmargin}{topmargin}{rightmargin}{bottommargin}{headheight}{headsep}{footheight}{footskip}
 \setmarginsrb{1.0in}     {0.8in}    {1.0in}      {.8in}        {.2in}      {0.3in}   {.2in}       {0.3in}

\pagestyle{fancy}
\footrulewidth 0.5pt
\lfoot{\small \today}
\cfoot{}
\rfoot{\small \thepage}
\lhead{IEEE Internet of Things Journal {\bf Paper:OoT-5520-2018  } - Authors' Response}


\newcommand {\mytitledproof}[1] {\medskip\noindent{\bf Proof (#1): }}
\def         \myproof           {\noindent{\bf Proof: }}
\def \myQED       {\hfill\QED \\ \vspace{-12pt} \\ \noindent}
\def \myQEDnotext {\hfill\QED}
\def \notation    {\medskip\noindent{\bf Notation: }}
\def \SINR        {{\rm SINR}}
\newcommand \Eh[1]    {{\mathbb E}_{\bbh}\left[ #1 \right]}
\newcommand \EhBig[1] {{\mathbb E}_{\bbh}\Big[ #1 \Big]}
\newcommand \Eht[1]   {{\mathbb E}_{\bbh(t)}\left[ #1 \right]}
\newcommand \Ep[1]    {{\mathbb E}_{m(\bbp(\bbh))}\left( #1 \right)}
\newcommand \Ehp[1]   {{\mathbb E}_{\bbh,\bbp(\bbh)}\left[ #1 \right]}

\def \distlamt        {\big\|\bblam(t)     - \bblam^* \big\|^2}
\def \distlamtns      {\big\|\bblam(t)     - \bblam^* \big\|}
\def \distlamtpone    {\big\|\bblam(t+1)   - \bblam^* \big\|^2}

\newcommand \Cif[1] {C_{if}\big(#1\big)}
%\newcommand \inlineCif[1] {C_{if}\big(#1\big)}
\def \Cift {\Cif{h_{if}(t),p_{if}(t)}}
\def \Cifh {\Cif{h_{if},p_{if}(\bbh)}}
\def \Cifhtpnot {\Cif{h_{if}(t),p_{if}}}
%\def \inlineCift      {\Cif{h_{if}(t),p_{if}(t)}}
%\def \inlineCifh      {\Cif{h_{if},p_{if}(\bbh)}}
%\def \inlineCifhtpnot {\Cif{h_{if}(t),p_{if}}}


\newcommand{\e}{\mathtt{e}}
\newcommand{\R}{\mathbb{R}} 
\newcommand{\N}{\mathcal{N}}
\def\forall{\text{for all\ }}
\def\N{n}
\def \sech {\text{sech}}

\def \hbS {{\bf \hat{S}}}

\def \mph         {m\big(\bbp(\bbh)\big)}
\def \mphstar         {m^{*}\big(\bbp(\bbh)\big)}
\def \mpht        {m\big(\bbp(\bbh(t))\big)}
\def \moph        {m_{0}\big(\bbp(\bbh)\big)}
\def \fphh        {\bbf_{1}\big(\bbp(\bbh);\bbh\big)}
\def \fphhu       {\bbf_{1}\big(\bbp(\bbh(u));\bbh(u)\big)}
\def \fphht       {\bbf_{1}\big(\bbp(\bbh(t));\bbh(t)\big)}
\def \fptht       {\bbf_{1}\big(\bbp(t);\bbh(t)\big)}
\def \fpuhu       {\bbf_{1}\big(\bbp(u);\bbh(u)\big)}
\def \fphtlamtht  {\bbf_{1} \big(\bbp(\bbh(t),\bblam(t));\bbh(t) \big)}
\def \fpohh    {\bbf_{1}\big(\bbp(\bbh);\bbh\big)}
\def \lagrangian  {\ccalL[\bblam,\bbx,\mph]}
\def \lagrangianforreviewer  {\ccalL[\bblam_{2},\bbx]}
\def \lagrangiansubzero  {\ccalL_{0}\left[\bblam,\bbx,\mph\right]}
\def \lagrangiansubzerot  {\ccalL_{0}\left[\bblam(t),\bbx,\mph\right]}
\def \lagrangiant {\ccalL\left[\bblam(t),\bbx, \mph\right]}
\def \dest {{\rm dest}}

\def\E{\mathbb{E}}

\begin{document}

\title{Control Aware Radio Resource Allocation in Low Latency Wireless Control Systems}
\author{Mark Eisen$^*$ \quad Mohammad M. Rashid$^\dagger$ \quad Konstantinos Gatsis$^*$ \\ Dave Cavalcanti$^\dagger$\quad Nageen Himayat$^{\dagger}$ \quad Alejandro Ribeiro$^*$ \\ $^{*}$University of Pennsylvania \quad $^{\dagger}$Intel Corporation} \maketitle                                    %%%%

\editor


We thank the associate editor and the reviewers for the time and effort invested in providing constructive suggestions. The requests most common among the reviewers concerned numerous points of clarification that add to the understandability and readability of the manuscript. Two reviewers requested additional comparisons in the numerical simulations. Reviewers also raised key questions regarding the placement of the proposed work in the context of ultra reliability low latency communications and control (URLLC).  A more detailed account of the major changes implemented in the manuscript follows. These changes are additionally highlighted in the manuscript in \red{red} font.

\begin{itemize}

\item[{\bf[C1]}] Reviewers 1 and 2 expressed concerns about the completeness the numerical simulations and felt that it is necessary to include a comparison against another control-aware scheduling method that is commonly used in wireless control systems. We believe this is a good point to be raised, and have made the following changes. Although there is not scheduling method to our knowledge that explicitly addresses the challenge of latency using control state information, we have adapted the broad ideas of event-triggered control to the low-latency scheduling framework developed here. We have added to the revised manuscript a comparison of the performance of the proposed CALLS method against an event-triggered low-latency scheduling alternative and show that the proposed method is able to outperform the standard event-triggered approach.


\item[{\bf[C2]}] Reviewer 1 has suggested some additional clarification and context of the work within the framework of URLLC. This was an important point to bring up and we thank the reviewer for this suggestion. To address this concern, we have revised the introduction of the manuscript to add important references on the topic of URLLC, as well as clarify the alternative approach taken in this work: namely, by using control system knowledge to derive \emph{adaptive} reliability rate targets, rather that fixed high or ultra reliability targets commonly used. These adaptive targets are shown to still allow for successful control of these systems and further enable low-latency by loosening the constraints on reliability.


\item[{\bf[C3]}] Reviewers raised numerous points of clarification regarding the various system models, such as the wireless control model and the fading channel model. We have individually addressed each concern by adding clarifying sentences to the revised manuscript.


\end{itemize}



%%%%%%%%%%%%%%%%%%%%%%%%%%%%%%%%%%%%%%%%%%%%%%%%%%%%%%%%%%%%%%%%%%%%%%%%
%%%   R   E   V   I   E   W   E   R    %%%%%%%%%%%%%%%%%%%%%%%%%%%%%%%%%
%%%%%%%%%%%%%%%%%%%%%%%%%%%%%%%%%%%%%%%%%%%%%%%%%%%%%%%%%%%%%%%%%%%%%%%%
\reviewer

We thank the reviewer for the comments, feedback, and questions. The reviewer raised concerns about the placement of the work within the URLL control and communication framework and literature. The reviewer also brings up points of clarification regarding the problem formulation. Each concern is addressed individually below.

\issue{ The authors need to recast the framework within ultra-reliable and low-latency control and communication.}
\answer{We thank the reviewer for this important comment and suggestion. We have revised the introduction of the manuscript substantially to contextualize the work being done here within the framework of URLLC. In doing so, we have added the following important references on the topic to the discussion in the introduction.
\begin{itemize}
\item M. Weiner, M. Jorgovanovic, A. Sahai, and B. Nikolie, “Design of a
low-latency, high-reliability wireless communication system for control
applications,” in Communications (ICC), 2014 IEEE International
Conference on. IEEE, 2014, pp. 3829–3835.
\item P. Popovski, J. J. Nielsen, C. Stefanovic, E. de Carvalho, E. Strom,
K. F. Trillingsgaard, A.-S. Bana, D. M. Kim, R. Kotaba, J. Park et al.,
“Wireless access for ultra-reliable low-latency communication: Principles
and building blocks,” IEEE Network, vol. 32, no. 2, pp. 16–23, 2018.
\item M. Bennis, M. Debbah, and H. V. Poor, “Ultra-reliable and lowlatency
wireless communication: Tail, risk and scale,” arXiv preprint
arXiv:1801.01270, 2018.
\item V. N. Swamy, S. Suri, P. Rigge, M. Weiner, G. Ranade, A. Sahai, and
B. Nikolic, “Cooperative communication for high-reliability low-latency
wireless control,” in Communications (ICC), 2015 IEEE International
Conference on. IEEE, 2015, pp. 4380–4386.
\item J. J. Nielsen, R. Liu, and P. Popovski, “Ultra-reliable low latency communication
using interface diversity,” IEEE Transactions on Communications,
vol. 66, no. 3, pp. 1322–1334, 2018.
\end{itemize}
In addition to this more comprehensive review of URLLC techniques, we have the introduction to highlight the fact that our control-aware scheduling approach can be seen as a means of defining an \emph{adaptive}-reliability framework for achieving low-latency communications, in contrast to fixed ultra reliability targets. This adaptive reliability framework considers the particular control system dynamics to determine the reliability necessary to keep all plants in good conditions. Our numerical simulations show how this alternative framework can improve the scale of low-latency wireless control systems by loosening reliability requirements where applicable. 
}

\issue{ It was unclear whether the focus is on communication for a given control or the other way around. Taht is currently it looks like the solution is based on sequential design.}
\answer{We thank the reviewer for brining up this point of clarification. The focus of this work is designing a scheduling protocol for a given system and control. As such, the control is assumed to be given, or chosen agnostic to the communication, while the communication is designed relative to both the system dynamics and choice of control. We have added a sentence to Section II to clarify this point. }

\issue{ The article would improve a lot if specific use cases are defined.}
\answer{We thank the reviewer for this suggestion. We have added a sentence to the introduction clarifying the use cases of particular interest in this work, namely the industrial control and factory settings, where there is significant interest in closing many low-latency control over wireless links.}

\issue{ Linear control was adopted (for tractability) but clearly the nonlinearity is needed}
\answer{The reviewer brings up a good point about the consideration of nonlinear control in additional the linear control that is discussed in this paper. We agree that a more complete discussion of this topic should include non-linear controllers. However, we first point out that many of the most commonly used controller in practice, e.g. LQR and PID control, are linear in nature and can be formulated in the manner discussed in the manuscript. Such controllers are used in both industrial and robotic systems. Therefore, we feel that the consideration of linear control not only allows for the precise and analytic derivation of PDR targets used in this work,  but also encompasses a large variety of practical control systems. The reviewer is correct, however, that more nonlinear control is also an important thing to consider. As the non-linearities further complicate the analysis performed here, we aim to study their use in this context in future work. We have added a sentence to this effect in the Conclusion section of the revised manuscript.   }

\issue{Reliability as not addressed however it is intertwined with latency}
\answer{This is a good point raised by the reviewer that deserves clarification. Reliability is a key component of a practical system that must be carefully managed in low latency systems. In our work, rather than aim for traditional, communication-based reliability targets, such as $0.999$ packet delivery rates, we consider the fact that the observed performance of a control system is more closely related to a \emph{control}-based reliability metric, such as stability or distance from origin. As discussed in Section III and IV, by defining reliability targets with respect to the Lyapunov cost function rather than with respect to a fixed packet delivery rate target, we are able to derive \emph{adaptive} packet delivery rate requirements that change with the current performance of each control system. When a control system is in a safe or stable range, it can remain in good performance with smaller packet delivery rates than a control system in a less desirable region. Thus, the proposed CALLS method aims to achieve reliability of control system performance, in other words an \emph{adaptive reliability}, in contrast to a standard high reliability framework. Such adaptive rates are critical in enabling low-latency transmissions in large-scale wireless settings, as observed our simulations in Section V. We point out that the proposed control-adaptive framework is thus presented as an alternative to high reliability scheduling schemes in settings in which the control systems are well-managed and defined, such as in the case of industrial controls. We have added a discussion of this to the revised manuscript.}

\issue{Authors did not cite recent and fundamental articles within URLLC:}
\answer{We thank the reviewer for pointing out these references. They have been added to the revised manuscript, as well as  included in an added discussion of URLLC in the introduction (Section I). The references added are listed in a previous response to this reviewer. }




%%%%%%%%%%%%%%%%%%%%%%%%%%%%%%%%%%%%%%%%%%%%%%%%%%%%%%%%%%%%%%%%%%%%%%%%
%%%   R   E   V   I   E   W   E   R    %%%%%%%%%%%%%%%%%%%%%%%%%%%%%%%%%
%%%%%%%%%%%%%%%%%%%%%%%%%%%%%%%%%%%%%%%%%%%%%%%%%%%%%%%%%%%%%%%%%%%%%%%%
\reviewer

We thank the reviewer for the comments, feedback, and questions. The reviewer raised concerns about the readability of the abstract and made some suggestions regarding the simulation results. Each concern is addressed individually below:

\issue{ Currently, it is not easy to follow the abstract. It should be revised. The contributions of the this paper should be highlighted.}
\answer{We thank the reviewer for raising this point. We have revised the abstract in the manuscript to be more clear and the highlight the specific contributions of the paper.}

\issue{  In simulation part, it would be great if the authors could compare the proposed CALLS method with other methods if possible, since there are also some papers discussing communication resource allocation for multiple control loops.}
\answer{We thank the reviewer for this suggestion. Indeed, an interesting comparison would be to compare the proposed CALLS method with other scheduling methods for wireless control systems. We first point out that existing control-aware scheduling methods do not explicitly attempt to enable low latency or are developed for the IEEE 802.11ax architecture. In any case, we can adapt the existing typical scheduling approach of \emph{event-triggered control} for the low latency setting. In such a scheduling scheme, a device is scheduled to transmit only if its estimated state goes beyond some threshold value. Should this event occur, the device is scheduled with a fixed, high reliability packet delivery rate using a low latency scheduling method (i.e. assignment-based scheduling). In a sense, this scheme combines the selective scheduling approach of CALLS method with the standard fixed PDR of URLLC. 

In the revised manuscript, we include a comparison against an event-triggered scheduling protocol for the balancing board ball experiments in Section V-B. As can be seen, event-triggered scheduling indeed outperforms the fixed PDR scheduling but does achieve the same scale in terms of number of devices supported as the CALLS method, which uses \emph{adaptive} PDR targets that make explicit use of control system dynamics to schedule devices. We lastly point out that these comparisons were also performed with the inverted pendulum system. However, the event triggered scheduling was not able to adapt to the faster dynamics of the inverted pendulum to control many devices, and such comparisons are therefore left out of the revised manuscript.}


%%%%%%%%%%%%%%%%%%%%%%%%%%%%%%%%%%%%%%%%%%%%%%%%%%%%%%%%%%%%%%%%%%%%%%%%
%%%   R   E   V   I   E   W   E   R    %%%%%%%%%%%%%%%%%%%%%%%%%%%%%%%%%
%%%%%%%%%%%%%%%%%%%%%%%%%%%%%%%%%%%%%%%%%%%%%%%%%%%%%%%%%%%%%%%%%%%%%%%%
\reviewer

We thank the reviewer for the comments, feedback, and questions. The reviewer raised points of clarification regarding the system setup and proposed additional comparisons to be made in the simulation results. Each concern is addressed individually below:

\issue{ Why do you only consider the AWGN channels?  How will your scheme performance when considering fading channels? Is the fading channel also a typical scenario for WiFi based in-door wireless control system?  }
\answer{We thank the reviewer for raising this point, as this aspect of the model is poorly worded in the previous version of the manuscript. We do indeed consider a fading channel to model the wireless channel conditions (specifically, IEEE Channel Model E). In our simulations, we use AWGN curves to evaluate packet error probabilities, but the curves are evaluated at the complete SNR values, which take into account the instantaneous fast fading channel conditions. In addition, we point out that IEEE Channel Model E is the current standard for measuring channel conditions in an indoor environment. More sophisticated models may be developed, but are outside the scope of this work. \blue{We have revised the manuscript to clarify these points regarding the wireless channel model.}}

\issue{In your system, how do you obtain accurate channel state information? E.g. for AWGN channels, you need to test the noise variance. Is your scheme sensitive to channel estimation errors?}
\answer{This is a good point raised by the reviewer. In our simulations, we consider that channel state information is obtained via pilot signals and there is no estimation errors. In practice, however, such estimation errors may indeed exist. We first point out that the computation of adaptive packet delivery rate targets in equation (23) does not depend upon channel state information, only the estimated control state information. Therefore, the main control-awareness component of the proposed scheduling protocol---namely Steps 3-6 in Algorithm 1--- is unaffected by inaccurate CSI estimation. Likewise, the assignment method-based scheduling in Step 8 does also not directly depend upon CSI information, as can be seen in the formulation of the assignment problem in equation (33). The only component that may be negatively impacted by estimation errors is Step 7,  the maximum MCS selection in equation (32). Here, the potential impact would be selecting an MCS that cannot meet the target PDR $\tdq_i$ due to an error in estimating $\bbh_{i,k}$.  Such an effect can be mitigated by selecting a more conservative MCS selection to account for channel estimation errors---e.g. transmitting at an MCS $\mu_{i,k}(\bbvarsigma) - 1$ rather than $\mu_{i,k}(\bbvarsigma)$. Such a conservative scheme would tradeoff latency to the benefit of reliability or robustness. A detailed numerical study of this effect is outside the scope of this work, however, and left as a study of future work. In the revised manuscript, we have added a remark to the end of Section IV discussing this point.   }

\issue{The proposed system shall be compared to more existed typical scheduling scheme employed in control system.}
\answer{We thank the reviewer for this suggestion. Indeed, an interesting comparison would be to compare the proposed CALLS method with other scheduling methods for wireless control systems. We first point out that existing control-aware scheduling methods do not explicitly attempt to enable low latency or are developed for the IEEE 802.11ax architecture. In any case, we can adapt the existing typical scheduling approach of \emph{event-triggered control} for the low latency setting. In such a scheduling scheme, a device is scheduled to transmit only if its estimated state goes beyond some threshold value. Should this event occur, the device is scheduled with a fixed, high reliability packet delivery rate using a low latency scheduling method (i.e. assignment-based scheduling). In a sense, this scheme combines the selective scheduling approach of CALLS method with the standard fixed PDR of URLLC. 

In the revised manuscript, we include a comparison against an event-triggered scheduling protocol for the balancing board ball experiments in Section V-B. As can be seen, event-triggered scheduling indeed outperforms the fixed PDR scheduling but does achieve the same scale in terms of number of devices supported as the CALLS method, which uses \emph{adaptive} PDR targets that make explicit use of control system dynamics to schedule devices. We lastly point out that these comparisons were also performed with the inverted pendulum system. However, the event triggered scheduling was not able to adapt to the faster dynamics of the inverted pendulum to control many devices, and such comparisons are therefore left out of the revised manuscript.}







\end{document}
