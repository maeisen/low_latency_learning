%% -*- Mode: LaTeX Memoir; tab-width: 4;

%% InvertedPendulum.tikz    Inverted Pendulum
%% Draw an inverted pendulum using PGF/TikZ
%% MAIN FILE: ../../TikZGallery.tex
%% Created by Dazhi Jiang, 2011-02-01 16:49:22 +0000 (Tue,  1 Feb 2011)
%% Copyright (c) 2011 Dazhi Jiang. All Rights Reserved. 

%% TextMate Settings
%!TEX root = ../../TikZGallery.tex

\newif\ifpendulumcolored
\pendulumcoloredtrue

\ifpendulumcolored

\tikzset{%
	interface/.style={
		% The border decoration is a path replacing decorator. 
		% For the interface style we want to draw the original path.
		% The postaction option is therefore used to ensure that the
		% border decoration is drawn *after* the original path.
		postaction={draw, decorate, decoration={border, angle=-45,
					amplitude=0.3cm, segment length=2mm}}},
	helparrow/.style={>=latex', blue, thick},
	% helparrow/.style={>=latex', draw=blue, fill=blue, very thick},
	helpline/.style={thin, black!90, opacity=0.5},
	force/.style={>=stealth, draw=blue, fill=blue, ultra thick},
	cart/.style={draw = black,%fill = black!40,%
				top color = green!5,%
				bottom color = green!40,%
				% pattern=horizontal lines gray,%
				},
	pendulum/.style={draw = black,% fill = black!25,%
					left color = red!10,%
					bottom color = red!40,%
					% pattern=horizontal lines gray,%
					},
	ground/.style={fill=brown!20},
	inner wheel/.style={fill=black!50},
	outer wheel/.style={thin,double = black!75,double distance = 0.6mm,black},
	wheel shadow/.style={thin,fill = black!70,path fading = south},
	xycoord/.style={<->, thick, black!50},
}

\else

\tikzset{%
	interface/.style={
		% The border decoration is a path replacing decorator. 
		% For the interface style we want to draw the original path.
		% The postaction option is therefore used to ensure that the
		% border decoration is drawn *after* the original path.
		postaction={draw, decorate, decoration={border, angle=-45,
					amplitude=0.3cm, segment length=2mm}}},
	helparrow/.style={>=latex', blue, thick},
	% helparrow/.style={>=latex', draw=blue, fill=blue, very thick},
	helpline/.style={thin, black!90, opacity=0.5},
	force/.style={>=stealth, draw=blue, fill=blue, ultra thick},
	cart/.style={draw = black!80,%fill = black!40,%
				top color = black!10,%
				bottom color = black!40,%
				% pattern=horizontal lines gray,%
				},
	pendulum/.style={draw = black!80,% fill = black!25,%
					left color = black!10,%
					bottom color = black!40,%
					% pattern=horizontal lines gray,%
					},
	ground/.style={fill=black!20},
	% ground/.style={fill=black!20},
	inner wheel/.style={fill=black!30},
	outer wheel/.style={thin,double = black!20,double distance = 0.5mm},
	wheel shadow/.style={thin,fill = black!70,path fading = south},
	xycoord/.style={<->, thick, black!50},
}

\fi

\def\ground{%
	\fill [ground] (0, 0) rectangle (80mm, -5mm);
	%\fill [ground] (49mm, 0) rectangle (80mm, -5mm);
	\draw [thick, black!80, interface] (0, 0) -- (80mm, 0);
	% \draw [|->, helparrow] (49mm, -9mm) -- ++(20mm, 0) node [right] {$\gls{x}$};
}

\def\cart{
	\filldraw [cart] (0,0) rectangle (30mm, 10mm);
	% \node at (15mm, 5mm) {$\gls{M}$};
	% \draw[->, force] (-15mm, 5mm) node [above] {$\gls{F}$} -- (0, 5mm);
	\draw[->, force] (-15mm, 5mm) node [above] {$u_{i,k}$} -- (0, 5mm);
	% \draw[->, force,] (40mm, -2.5mm) node [above, text = blue] {${\gls{b}}\textcolor{blue}{\dot{\gls{x}}}$} -- (26.5mm, -2.5mm);
	% \draw[->, force,] (40mm, -2.5mm) node [above, text = blue] {$\gls{b}\gls{xspeed}$} -- (26.5mm, -2.5mm);
	% \draw[->, force] (-15mm, 5mm) node [above] {$\vv{\gls{F}}$} -- (0, 5mm);
	% \draw[->, force] (-15mm, 5mm) node [above] {$\vv{F}$} -- (0, 5mm);
}

\def\pendulum{%
	\filldraw [pendulum] (-0.8mm, 0) rectangle (0.8mm, 20mm);
}

\def\joint{%
	\filldraw [%thick,%
		draw = black!80,%
		fill = white,%
		% pattern=horizontal lines gray,%
		] (0, 0) circle (1mm);
}

\def\wheel{%
	\fill [wheel shadow] (0, 0) circle (1.75mm);
	\begin{scope}
		\clip (0, 0) circle (1.75mm);
		\fill [inner wheel] (0, -1mm) circle (2mm);
	\end{scope}
	\fill [fill=black!90] (0, 0) circle (0.5mm);
	\draw [outer wheel] (0, 0) circle (2mm);
}

\def\xycoord{%
	\draw [xycoord] (8mm,0) node [below right] {$x$} -- (0,0) -- (0,8mm) node [above left] {$y$};
}

%\tikzsetnextfilename{PIDBode}
\begin{tikzpicture}[> = latex',%
		scale = 1,%
		text = blue,%
		execute at end picture = {
			\begin{pgfonlayer}{background}
				%% * Compute a few help coordinates
				% \coordinate (bl) at (-40mm, -24mm);
				% \coordinate (tr) at (58mm, 45mm);
				\coordinate (bl) at ($(current bounding box.south west) + (-5mm, -5mm)$);
				\coordinate (tr) at ($(current bounding box.north east) + (+5mm, +5mm)$);
%				\draw [very thin, fill=white, %
%					  rounded corners = 1.5pt, %
%					  draw=black!50, %
%					  % decorate, %
%					  % decoration={random steps,segment length=3pt,amplitude=1pt}, %
%					  drop shadow={fill=black!30}, %
%					  % general shadow={fill=black!10, shadow scale=1.01}, %
%					  % dashed, %
%					  ]
%						% (inner box.south west) + (-5mm, -5mm) rectangle 
%						% (inner box.north east) + (5mm, 5mm);
%						(bl) rectangle (tr);
				%\useasboundingbox (bl) + (-1mm, -1mm) -- (tr) + (1mm, 1mm);
			\end{pgfonlayer}
		}
		]
		\begin{scope}
			\wheel
		\end{scope}
		\begin{scope}[xshift=18mm]
			\wheel
		\end{scope}
		\begin{scope}[xshift=-31mm, yshift=-2.4mm]
			\ground
		\end{scope}
		\begin{scope}[shift = {(-6mm, 2.4mm)}]
			\cart
		\end{scope}
		\begin{scope}[shift = {(9mm, 12.4mm)}]
			\draw [helpline] (0, -20mm) -- (0, 25mm);
			\draw [helpline] (110:20.5mm) -- (110:25mm);
			\draw [|->, helparrow] (0, 23mm) 
				arc [radius=23mm, start angle=90, delta angle=20] ;
			% \node at (97:25mm) {$\gls{thetaa}$};
			\node at (97:25mm) {$\theta_{i,k}$};
			% \node [above right] at (50:8.5mm) {$\gls{theta}$};
				% node [left] {$\gls{m}$, $\gls{I}$};
		\end{scope}
		\begin{scope}[shift = {(9mm, 12.4mm)}, rotate=20]
			\pendulum
		\end{scope}
		\begin{scope}[shift = {(9mm, 12.4mm)}]
			\joint
		\end{scope}


\end{tikzpicture}