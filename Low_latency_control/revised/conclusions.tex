In this paper we proposed a novel control-communication co-design approach to solving the radio resource
allocation problem for time-sensitive wireless control systems. Given a channel state and control state, we mathematically derive a minimum packet delivery
rate a device must meet to maintain a control-orientated target,
as defined by a stability-inducing Lyapunov function. By
dynamically assigning variable packet delivery rate targets
to each device based on its current conditions, we are able to
more easily meet feasibility requirements of a latency-constrained
wireless control problem and maintain stability
and strong performance. We perform simulations on numerous
well-studied low-latency control problems to demonstrate the
benefits of using the control-aware approach, which can include
a 2x gain on number of devices that can be supported.

\red{The results presented in this paper suggest an interesting
potential for control-aware resource allocation and scheduling, particularly in low-latency industrial systems. By considering the control-specific targets such as maintaining
stability or an error margin, we observe that the standard
high reliability targets considered in URLCC (e.g. packet delivery rates $\geq 0.999$) can in some cases be substantially stricter than
necessary for adequate performance. Wireless control systems
with sufficiently slow dynamics can be kept stable with much
lower packet delivery rates, which in turn make low-latency
communications more achievable.} Furthermore, in realistic industrial systems there will be many heterogeneous devices being controlled, whose variation in communication needs is well-served by control-aware adaptivity proposed in this paper. This suggests the potential for wireless communications to be adopted using a smart control-communication co-design approach even
while ultra-reliable wireless system technology remains under development.